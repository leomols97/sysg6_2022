\section{Intégrer une source}
\subsection{Configuration de lstlisting}
% Le manuel se trouve dans le howto ou dans CTAN sous le nom de listings.pdf
La commande lstset permet de fixer la présentation des sources. Il n'est pas conseillé d'utiliser utf8 pour les sources si le document n'est pas en utf8.
\lstset{language={},%C,Assembleur, TeX, tcl, basic, cobol, fortran, logo, make, pascal, perl, prolog, {}
	literate={â}{{\^a}}1 {ê}{{\^e}}1 {î}{{\^i}}1 {ô}{{\^o}}1 {û}{{\^u}}1
		 {ä}{{\"a}}1 {ë}{{\"e}}1 {ï}{{\"i}}1 {ö}{{\"o}}1 {ü}{{\"u}}1
		 {à}{{\`a}}1 {é}{{\'e}}1 {è}{{\`e}}1 {ù}{{\`u}}1 
		 {Â}{{\^A}}1 {Ê}{{\^E}}1 {Î}{{\^I}}1 {Ô}{{\^O}}1 {Û}{{\^U}}1
		 {Ä}{{\"A}}1 {Ë}{{\"E}}1 {Ï}{{\"I}}1 {Ö}{{\"O}}1 {Ü}{{\"U}}1
		 {À}{{\`A}}1 {É}{{\'E}}1 {È}{{\`E}}1 {Ù}{{\`U}}1,
	commentstyle=\scriptsize\ttfamily\slshape, % style des commentaires
	basicstyle=\scriptsize\ttfamily, % style par défaut
	keywordstyle=\scriptsize\rmfamily\bfseries,% style des mots-clés
	backgroundcolor=\color[rgb]{.95,.95,.95}, % couleur de fond : gris clair
	framerule=0.5pt,% Taille des bords
	frame=trbl,% Style du cadre
	frameround=tttt, % Bords arrondis 
	tabsize=3, % Taille des tabulations
%	extendedchars=\true, % Incompatible avec utf8 et literate
	inputencoding=utf8,
	showspaces=false, % Ne montre pas les espaces 
	showstringspaces=false, % Ne montre pas les espaces entre ''
	xrightmargin=-1cm, % Retrait gauche 
	xleftmargin=-1cm, % Retrait droit
	escapechar=@}  % Caractère d'échappement, permet des commandes latex dans la source
% -----------------------------------------------------
\subsection{Intégrer une source dans le texte}
\begin{lstlisting}
/*---------------------------------------------------------------------------------------
NOM      : Exemple.c
CLASSE   : Applications - Latex - Illustration
OBJET    : Sert d'exemple pour inclure une source en latex
         : Dans ce ces, ce programme affiche Hello
HOWTO    : gcc Exemple.c -o Exemple; ./Exemple
AUTEUR   : J.C. Jaumain, le 3/11/2010
BUGS     :  /
REMARQUE : Impose lstset {escapechar=@\symbol{64}@} pour l'interprétation des balises latex
----------------------------------------------------------------------------------------*/
main() {
	int i;  // Pour récupérer le nombre de caractères écrits
	tab[10] Buffer='Hello'; // Le buffer
	i=write(1,Buffer,5); // La @$\frac{1}{2}$@ du buffer
	exit(0);
}
\end{lstlisting}

L'intérêt de cette technique est de figer le source et d'avoir un document autonome

\subsection{Intégrer une source d'un fichier}

\lstinputlisting{Exemple.c}

L'intérêt de cette technique est d'avoir un source toujours "à jour".



