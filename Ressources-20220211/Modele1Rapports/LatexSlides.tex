\slideCaption{\large{\copyright J.C. Jaumain -- Prosper -- }}

\begin{slide}{Objectifs}
L'objectif de ce travail est de fournir une base pour celui qui doit afficher un exposé sous forme de slides pour mon cours. Il commence par une suite d'exemples suffisante pour pouvoir établir un premier exposé correct. 
\end{slide}
%-------------------------------------------------------------------------------------
\begin{slide}{Une première approche très simple}
Vous découpez votre travail en slides que vous encadrez par les balises begin~slide et end~slide
\end{slide}
%-------------------------------------------------------------------------------------
\begin{slide}{Latex}
Pratiquement, presque tout ce que vous utilisez dans l'écriture d'un rapport peut être utilisé dans les slides. Le travail le plus important est la découpe logique de votre exposé en slides.
\end{slide}
%-------------------------------------------------------------------------------------
\begin{slide}{Quelques trucs faciles}
\begin{list}{-}{}
\item documentclass wj donne des slides très simples avec un léger filet. Il en existe une trentaine d'autres (voir /usr/share/texmf/tex/latex/prosper). Vous pouvez choisir un autre style sans oublier qu'il faut imprimer les slides. N'utilisez pas des styles avec un fond coloré ! (ou remettez wj pour l'impression)
\item Il est possible de rendre vos slides dynamiques. Voir le mode d'emploi de prosper. Pour un premier travail, cela n'est pas utile.
\item prosper et beamer sont deux solutions pour créer vos slides en latex. L'évolution de prosper (powerdot) paraît la plus intéressante. 
\item Si vous utilisez beamer, vérifier que l'impression de vos slides est correcte.
\end{list}
\end{slide}
%-------------------------------------------------------------------------------------
\begin{slide}{Annexes}
Vous trouvez dans le casier, un répertoire LATEX contenant :
\begin{list}{-}{}
\item Mslides.tex : le document latex maître
\item LatexSlides.tex : ce document latex
\item goSlides : un script qui permet de compiler Mslides.tex, sans argument.
\end{list}
\end{slide}
%-------------------------------------------------------------------------------------
\begin{slide}{Question}
~\\[3.5cm]
 \begin{center}
 \fbox{\Huge{Avez-vous des questions ?} }
 \end{center}
\end{slide}
%-------------------------------------------------------------------------------------