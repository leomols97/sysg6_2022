\section{Programmation}

% \subsection{Variables}   REPRENDRE LES SLIDES
% 
% Deux sortes de variables : les entiers et les dimensions. Les dimensions sont toujours accompagnées d'une unité. (mm, cm, pt,...)
% 
% \subsubsection{variables}
% \begin{itemize}
% \item Création : 
% \item Assignation :
% \item Incrémentation :
% \item Tests :
% \end{itemize}
% 
% \subsubsection{variables prédéfinies}
% \begin{itemize}
% \item 
% \item 
% \item 
% \item 
% \item 
% \item 
% \item 
% \item 
% \item 
% \item 
% \item 
% \end{itemize}
% 
% \subsubsection{dimensions}
% \begin{itemize}
% \item Création : 
% \item Assignation :
% \item Incrémentation :
% \item Tests :
% \end{itemize}
% 
% \subsubsection{dimensions prédéfinies}
% \begin{itemize}
% \item 
% \item 
% \item 
% \item 
% \item 
% \item 
% \item 
% \item 
% \item 
% \item 
% \item 
% \end{itemize}
% 
% 
% \section{contrôles}
% 
\section{commandes}

\begin{verbatim}
\newcommand{\NomCmd}[argc][def1]
{ Commandes où les arguments s'appellent #1 #2...
}
ou \renewcommand...
\end{verbatim}
où
\begin{itemize}
\item NomCmd est le nom donné à la commande 
\item argc est le nombre d'argument, 0 par défaut
\item def1 est une valeur par défaut pour le premier argument.
\item newcommand pour définir une nouvelle commande dont le nom n'existe pas 
\item renewcommand pour redéfinir une commande dont le nom existe déjà 
\end{itemize}

\subsection{Exemples:}

\newcommand{\EtatCivil}[3]{\textbf{#1} #2 #3}
\EtatCivil{M.}{Albert}{Einstein}\\
\renewcommand{\EtatCivil}[3][M.]{\textbf{#1} #2 #3}
\EtatCivil{Albert}{Einstein}\\
\EtatCivil[Mme]{Julio}{Curie}\\

% \section{environnement}    NE FONCTIONNE PAS !!!!
% 
% \begin{verbatim}
% \newenvironment{NomEnv}[argc][def1]
% {sequenceDebut}{SectionFin} où les arguments s'appellent #1 #2...
% \end{verbatim}
% où
% \begin{itemize}
% \item NomEnv est le nom donné à l'environnement 
% \item argc est le nombre d'argument, 0 par défaut
% \item def1, optionnel, est une valeur par défaut pour le premier argument.
% \item newenvironment pour définir un nouvel environnement dont le nom n'existe pas 
% \end{itemize}
% 
% \subsection{Exemples:}
% 
% \newenvironment{MonTitre}[1]{\
% begin{center}\textbf{#1}\\}{\end{center}}
% 
% \begin{MonTitre}{Première ligne}
% Ligne deuxième
% \end{MonTitre}


