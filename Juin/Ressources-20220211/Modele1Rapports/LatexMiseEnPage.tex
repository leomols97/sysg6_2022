\section{Mise en page}

\subsection{Marges...}

Une série de variables définissent la mise en page. En utilisant les packages [francais]{layout} et {fullpage}, on peut utiliser la commande layout qui permet d'ajouter une page qui dessine la présentation d'une page et les noms des variables assignées.

\layout

On peut ensuite modifier ce que l'on souhaite :
\begin{verbatim}
% Modification des marges ------------------------------
\oddsidemargin -4mm 	% Marge de gauche -4mm
\textwidth 17cm 	% Largeur de gauche = 17cm
\textheight 22cm 	% Hauteur du texte = 22cm
\parindent 0cm		% Pas d'indentation de paragraphe
% -----------------------------------------------------
\end{verbatim}

\subsection{Niveaux}

Vous avez droit à la structure

\begin{list}{$\bullet$}{}
\item part avec saut de page
\item chapter : niveau 0
\item section : niveau 1
\item subsection : niveau 2
\item subsubsection : niveau 3
\item paragraph : niveau 4
\item subparagraph : niveau 5
\end{list}

La variable secnumdepth permet de limiter la numérotation des niveaux. Par exemple, la valeur 5 permet d'avoir une numérotation du style 1.2.3.2.1.2 pour le subparagraph\footnote{De la même façon, on peut limiter le nombre de niveaux affichés dans une table des matières avec la variable tocdepth }. 
(Il est toujours possible d'insérer une note de bas de page avec \verb+\footnote+)

\subsection{Cadres}

Il est possible d'encadrer un mot avec \fbox{box}

%\shabox{On peut aussi encadrer tout un paragraphe avec l'environnement shabox à condition de déclarer le package shadow avant le début du document}.
